\documentclass[./dissertation.tex]{subfiles}
\begin{document}
    \chapter{Conclusion}
In conclusion, this PhD thesis may have not made significant and practical contributions to the field, but by addressing critical gaps in knowledge, developing innovative methodologies, and providing valuable insights paved the way for further research. The Field of Automatic Certification and safety assessment is rather open field, joinin the specialized conferences and symposiums it is clear that the community it is, yes willing, but not read to yield the responsability of such and analyis to an automated method. Nevertheless the rigorous theoretical framework presented in the thesis, which integrates and builds upon previous literature, serves as a solid foundation for the empirical analysis future PhD will continue, under different spotlights to enrich this research, until finally it will be the state of the art for the relibility assessment of electronic components. In any case, the extensive data collection, meticulous experimental design, and robust validation of the proposed models have allowed me to have a more comprehensive understanding of the subject matter, thus enhancing the existing body of knowledge and allowing a completlely different vision of the field.

%%%%%%%%%%%%%%%%%%%%%%%%%%% NEW PART

The research conducted in this thesis has successfully adapted and extended the fault tree-based extraction of failure metrics, which has conventionally been applied to macroscopic electromechanical systems, to the realm of automotive SoCs. The current state-of-the-art in automotive SoCs involves largely manual metric extraction and expert dependence, whereas the verification of countermeasure effectiveness is typically performed through targeted fault injection on specific sub-parts of the system or irradiation under a particle beam.

The results given by the first part of the thesis involved the development of a methodology capable of the extraction of metrics at the block level using two different methods: an analytical method based on probabilities and an experimental method based on fault injection. Both gave the expected results, showing that it is possible to perform a much more effective metric extraction campaign at block level followed by a careful reconstruction. The focus of this research was not to create new probability codes or fault injection tools, but rather to establish a methodology for employing these techniques in the context of an SoC to obtain the desired data.

The second part of the thesis addressed the composition of data obtained at the functional block level to derive the ISO26262 metrics at the system level (SoC). The developed composition method was tailored to the unique characteristics of SoCs, such as their communication systems, real-time calculations, and fault models imposed by the ISO26262 standard. The focus was put on the inter layer propagation of faults and the ability in this case once again of building a cost effective methodology, that in this case has had a much larger impact on costs and time given the nature of the classical simulation campaigns.

Finally, the interest that this thesis has obtain in the field of harsh environment electronics as it could be the High Energy Phisics lab as CERN, is a coronation for the efforts put in these three years in the development of this research thread. Even though not completed yet, the campaigns on the PicoRVino SoC could be crucial in steering the path of the verification for these kind of components.

Through this research journey, the thesis has contributed to bridging the gap between traditional fault tree-based methods and their application to the automotive SoC domain. The findings have not only enhanced the existing literature but also provided practical implications for stakeholders, policymakers, and practitioners involved in the development and certification of automotive SoCs.

As with any research endeavor, this thesis has also identified areas for future exploration and improvement. Future researchers can build upon the foundations laid in this work to further refine the proposed methodology, expand its applicability to other domains, and develop innovative techniques for extracting and analyzing reliability metrics in complex systems.

In conclusion, this thesis could marks a significant milestone in the field of automotive SoC reliability. The theoretical framework, extensive empirical analysis, and robust validation of the proposed models have provided a comprehensive understanding of the subject matter, paving the way for further research and advancements. The interdisciplinary nature of the study has fostered collaborations and synergies among different domains, encouraging a more holistic approach to problem-solving. As the field continues to evolve, the knowledge generated by this research will serve as a stepping stone for further innovations and the development of more efficient and reliable automotive systems.

This thesis is a testament to the power of collaboration, perseverance, and the relentless pursuit of knowledge in the service of a better understanding of our world. As I embark on the next chapter of my academic journey, I carry with me the lessons learned, the skills acquired, and the passion for discovery that has been nurtured throughout the course of my doctoral studies. I am committed to using my research as a catalyst for positive change and to continuing my lifelong pursuit of knowledge in the service of society.
\end{document}
