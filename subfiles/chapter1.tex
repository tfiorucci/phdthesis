\documentclass[./dissertation.tex]{subfiles}
\begin{document}


\chapter{Failure Mode and Effect Analysis FMEA}

\section{Introduction}
\subsection{General}
FMEA and FMECA are important techniques for a reliability assurance programme.They can be applied to a wide range of problems which may occur in technical systems, and can be carried out in varying degrees of depth, or modified, to suit a particular purpose. The analysis is carried out in a limited way during the conception, planning, and definition phases and more fully in the design and development phase. It is however important to remember that the FMEA is only part of a reliability and
maintainability programme which requires many different tasks and activities. FMEA is an inductive method of performing a qualitative system reliability or safety analysis from a low to a high level. A thorough understanding of the system under analysis is essential prior to undertaking FMEA. Functional diagrams and other system drawings are normally necessary for this understanding. Reliability block diagrams, fault trees and/or state diagrams are then usually derived from these in order to
carry out the analysis. In many instances the block diagram descriptions and block diagram failure descriptions are included in the FMEA format. Separate diagrarns will be needed for the
following:
\begin{enumerate}
    \item The way in which different criteria for system faiulre are determined;
    \item Degradation of function or reduction in assurance of function;
    \item Alternative operational phases
\end{enumerate}

\subsection{Purpose of the Analysis}
The reasons for undertaking FMEA (or FMECA) may include the following:

\begin{itemize}
    \item to identify those failures which have unwanted effects on system operation, e.g. safety critical failures;
    \item to satisfy contractual conditions that an FMEA should be completed;
    \item where appropriate, to quantify the reliability and/or safety of the system;
    \item to allow improvements of the system's reliability and/or safety (e.g. by design or quality assurance action)
    \item to produce aids to fault diagnosis;
    \item to allow improvement of the system's maintâinability (by highlighting areas of risk or non-conformance for maintainability).
\end{itemize}

ln view of these reasons the objectives of an FMEA (or FMECA) may include the following: 

\begin{enumerate}
    \item a comprehensive identification and evaluation of all the unwanted effects within the defined boundaries of the system being analysed, and the sequences of events brought about by each identified item failure mode, from whatever cause, at various levels of the system's functional hierarchy;
    \item the determination of the significance (or criticality) of each failure mode with respect to the system's correct function or performance and the impact on the reliability and/or safety of the process concerned;
    \item a classification of identified failure modes according to relevant characteristics, including detectability, diagnosability, testability, item replaceability, compensating and operating provisions (repair, maintenance, logistics, etp.);
    \item an estimation of measures of the significance and probability of failure.
\end{enumerate}


\subsection{Basic Principles of FMEA}
The following concepts are essential to FMEA:

\begin{enumerate}
    \item breakdown of the system into 'elements';
    \item a diagram of the system's functional structure and identification of the various data which are needed to perform the FMEA;
    \item the failure mode concept (a part may have several failure modes or a failure mode may involve several parts);
    \item identification of new physical features or new requirements;
    \item the criticality concept and the measure to be used (if criticality analysis is required).
\end{enumerate}

Further, it is essential to specify the existing links between the FMEA (and the FMECA) and other qualitative (and quantitative) analytical methods within the overall reliability programme.  Very few designs are wholly new. Most are to some extent developments of old designs. FMEA should use the information on existing systems and draw attention to the need for tests, etc. for the new parts.  







\end{document}
