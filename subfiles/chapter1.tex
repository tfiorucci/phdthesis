\documentclass[./dissertation.tex]{subfiles}
\begin{document}


\chapter{Failure Mode and Effect Analysis FMEA}

\section{Introduction}
\subsection{General}
FMEA and FMECA are important techniques for a reliability assurance programme.They can be applied to a wide range of problems which may occur in technical systems, and can be carried out in varying degrees of depth, or modified, to suit a particular purpose. The analysis is carried out in a limited way during the conception, planning, and definition phases and more fully in the design and development phase. It is however important to remember that the FMEA is only part of a reliability and
maintainability programme which requires many different tasks and activities. FMEA is an inductive method of performing a qualitative system reliability or safety analysis from a low to a high level. A thorough understanding of the system under analysis is essential prior to undertaking FMEA. Functional diagrams and other system drawings are normally necessary for this understanding. Reliability block diagrams, fault trees and/or state diagrams are then usually derived from these in order to
carry out the analysis. In many instances the block diagram descriptions and block diagram failure descriptions are included in the FMEA format. Separate diagrarns will be needed for the
following:
\begin{enumerate}
    \item The way in which different criteria for system faiulre are determined;
    \item Degradation of function or reduction in assurance of function;
    \item Alternative operational phases
\end{enumerate}

\subsection{Purpose of the Analysis}
The reasons for undertaking FMEA (or FMECA) may include the following:

\begin{itemize}
    \item to identify those failures which have unwanted effects on system operation, e.g. safety critical failures;
    \item to satisfy contractual conditions that an FMEA should be completed;
    \item where appropriate, to quantify the reliability and/or safety of the system;
    \item to allow improvements of the system's reliability and/or safety (e.g. by design or quality assurance action)
    \item to produce aids to fault diagnosis;
    \item to allow improvement of the system's maintâinability (by highlighting areas of risk or non-conformance for maintainability).
\end{itemize}

ln view of these reasons the objectives of an FMEA (or FMECA) may include the following: 

\begin{enumerate}
    \item a comprehensive identification and evaluation of all the unwanted effects within the defined boundaries of the system being analysed, and the sequences of events brought about by each identified item failure mode, from whatever cause, at various levels of the system's functional hierarchy;
    \item the determination of the significance (or criticality) of each failure mode with respect to the system's correct function or performance and the impact on the reliability and/or safety of the process concerned;
    \item a classification of identified failure modes according to relevant characteristics, including detectability, diagnosability, testability, item replaceability, compensating and operating provisions (repair, maintenance, logistics, etp.);
    \item an estimation of measures of the significance and probability of failure.
\end{enumerate}


\subsection{Basic Principles of FMEA}
The following concepts are essential to FMEA:

\begin{enumerate}
    \item breakdown of the system into 'elements';
    \item a diagram of the system's functional structure and identification of the various data which are needed to perform the FMEA;
    \item the failure mode concept (a part may have several failure modes or a failure mode may involve several parts);
    \item identification of new physical features or new requirements;
    \item the criticality concept and the measure to be used (if criticality analysis is required).
\end{enumerate}

Further, it is essential to specify the existing links between the FMEA (and the FMECA) and other qualitative (and quantitative) analytical methods within the overall reliability programme.  Very few designs are wholly new. Most are to some extent developments of old designs. FMEA should use the information on existing systems and draw attention to the need for tests, etc. for the new parts.  



\section{Procedure}

\subsection{General}
The wide variation in complexity of system designs and applications may require the development of highly individualized FMEA procedures consistent with the information available. Traditionally, there have been wide variations in the manner in which FMEA is conducted and presented. However, the analysis is usually done in a standard manner and presented on a worksheet that contains a core of essential information which can be developed and extended to suit the particular system or project to which it is applied. A typical example of a worksheet is shown in Figure 1.

The procedure consists of the following four main stages:

\begin{enumerate}
\item Preparatory definition of the system including the design, functional, operational, maintenance, and environmental requirements;
\item Establishment of the basic principles and purposes of the FMEA and the form of its presentation;
\item Carrying out the FMEA using the appropriate worksheet designed according to (a) and (b);
\item Reporting of the complete analysis including any conclusions and recommendations made.
\end{enumerate}

A more detailed consideration of the information needed is given in Section 2.4.2.

\subsection{Preparation}
At the commencement of an analysis, the following preparations should be made:
\begin{enumerate}
\item The analyst should have available the information listed in Section 2.4.2.2 to 2.4.2.7 that clearly defines the system to be analyzed.
\item It will usually be necessary for the analyst to translate the information into some form of functional, hierarchical, or reliability block diagrams. An example of a functional diagram is shown in Figure 2. This diagram shows how the failure effects at the part level form the failure modes at the module level, the failure effects at the module level form the failure modes at the subsystem level, and so on. Such a representation of the system should explicitly identify the system's functional structure, the system boundary, and the inputs and outputs crossing that boundary. Further information is given in Section 2.4.2.8 to 2.4.2.10.
\end{enumerate}

\subsection{FMEA principles}
The following principles should be applied:
\begin{enumerate}
\item Define clearly the purposes and uses of the FMEA as indicated in Section 2.1.2.
\item Establish and define the relationships with other forms of reliability analysis with which the FMEA may subsequently be integrated. (See Section 2.3.5.)
\item Define the scope of the FMEA in relation to the functional structure and hierarchical structure of the system as described by the block diagrams referred to in Section 2.4.2.10. It is essential to define the lowest level in the system's hierarchical structure at which the analysis will start. The guidance given in Sections 2.3.4, 2.4.1, and 2.4.2.8 is especially important for this task.
\item Define the format of the FMEA worksheet to suit the project requirements. The core information considered essential is as follows:
\end{enumerate}

\begin{enumerate}
\item The name of the item in the system being analyzed;
\item Function performed by the item;
\item Identification number of the item;
\item Failure modes of the item;
\item Failure causes;
\item Failure effects on the system;
\item Failure detection methods;
\item Compensating provisions;
\item Severity of effects;
\item Remarks.
\end{enumerate}
Other information required for the particular system and project needs to be defined by the analyst according to the purposes of the



\section{Analysis}
It is worth to underline that, even though the scope of this thesis is to establish a new methodologyfor automated FMEA in both hardware and software systems, there is the absolute need to comply with what is today the state of the art for FMEA and the guidelines to be followed to bring the metrics extracted during the analysis to certification. It is for this reason that the following section will present the traditional method applied today to all the electromechanical systems under evaluation. Bare in mind that the totality of the procedure described below is man driven.

The usual requirement and purpose of an FMEA isto identify the effect of oJl failure modes of allconstituent items at the lowest level in the system.Tb achieve this the worksheet should be used inthe following manner:

\begin{enumerate}

	\item Identify all items in the system or subsystem,each of which is to have its failure modes andeffects analysed. The system of identification byname and number should be such that no itemswill be omitted.
	\item Select the first item for analysis and enter theitem name and identification number in theappropriate columns of the worksheet.Determine the function of that item in thesystem and enter that on the worksheet.
	\item Deduce all the possible failure modes of theitem due to any possible cause and individuallyenter these modes on the worksheet
	\item  Postulate the most likely failure causes foreach failure mode of the item and enter these onthe worksheet. It will usually not be possible to consider allpossible causes because the range is so vast, butthe most significant with regard to the item, thefailure mode and the application should beidentified.
	\item Deduce the effects of the failure on thesubsystem and system, as determined by thescope of the FMEA
	\item  Complete the remaining columns of theworksheet for the first failure mode of the firstitem.
	\item Repeat 3 to 5 for all failure moides of the first item
	\item Repeat 2 to 6 for all other items
\end{enumerate}


\subsection{Multiple Stages}
If the FMEA is to be done in stages that eachrelate to separate Ievels in the system's hierarchicalstructure, the failure effects from the lower level become the failure modes at the next level up. The analysis should then proceed as follows.

\begin{enumerate}
	\item Identify the lower level FMEAs that areappropriate for the next stage in the systemFMEA according to the system's hierarchicalstructure defined by the block or functionaldiagrams (see 2.2,2(b)). Where appropriate alsoinclude items defined as being at the lowest levelin that part of the system structure
	\item Perform the FMEA for each failure of eachitem at this higtrer level in the system stmctureas given in the previous section.
	\item repeat the two above steps for any further higher levels in the system structure.
\end{enumerate}
\subsection{Worksheet reccomendations}

The last worksheet entry should give any pertinent remarks to clarify other entries. Possible future actions such as recommendations for design improvements may be recorded and then amplified in the report. This column may also include the following:

\begin{enumerate}
\item[(a)] any unusual conditions;
\item[(b)] effects of redundant element failures;
\item[(c)] recognition of specially critical design features;
\item[(d)] any remarks to amplify the entry;
\item[(e)] references to other entries for sequential failure analysis;
\item[(f)] significant maintenance requirements;
\item[(g)] dominant failure causes;
\item[(h)] dominant failure effects;
\item[(i)] decisions taken, e.g. at design review.
\end{enumerate}


The report on the FMEA (or FMECA) may be included in a wider study or may stand alone. In neither case, the report should include a summary and a detailed record of the analysis and the block or functional diagrams which define the system structure. The report should also contain a list of the drawings (including issue status) on which the FMEA is based.

The summary should contain a brief description of the method of analysis and the level to which it was conducted, the assumptions and the ground rules. In addition, it should include listings of the following:

\begin{enumerate}
\item recommendations for the attention of designers, maintenance staff, planners, and users;
\item failures which, when initially occurring alone, result in serious effects;
\item failures which have no effect;
\item design changes which have already been incorporated as a result of the FMEA (or FMECA).
\end{enumerate}



\begin{figure}
        \includegraphics[width=\linewidth]{subfiles/imgs/FMEA_table.png}
  \caption{Example of FMEA Table}
        \label{fig:fmea_table}
\end{figure}


\begin{figure}
        \includegraphics[width=\linewidth]{subfiles/imgs/fmea_relationship_failure_mode_effects.png}
  \caption{Criticality Grid}
        \label{fig:criticality_grid}
\end{figure}


\subsection{Report on Analisys}

The report on Failure Modes and Effects Analysis (FMEA) or Failure Modes, Effects and Criticality Analysis (FMECA) can be either a standalone document or a part of a broader study. In either case, the report should contain a summary and a detailed record of the analysis along with block or functional diagrams that define the structure of the system. Furthermore, the report should include a list of drawings (with issue status) on which the FMEA is based.

The summary section of the report should provide a brief explanation of the analysis method, the level to which it was carried out, the assumptions made, and the ground rules followed (see section 2.4.L). The summary should also include lists of the following:

\begin{enumerate}
\item Recommendations for designers, maintenance staff, planners, and users;
\item Failures that, when occurring alone, have significant consequences;
\item Failures that have no impact; and
\item Design changes that were made as a result of the FMEA (or FMECA).
\end{enumerate}



\section{Application}
\subsection{Field of application}
FMEA is a method that is primarily adapted to the study of material and equipment failures and that can be applied to categories of systems based on different technologies (electrical, mechanical, hydraulic, etc.) and combinations of technologies. FMEA should also include the consideration of software and human performance where these are relevant to the reliability of the system. An FMEA can be a study for general use or it may be specific to particular pieces of equipment, to systems or to projects as a whole.

\subsection{Application within a project}
The user should determine how and for what purposes he uses FMEA within his own technical discipline. It may be used alone or to complement and support other methods of reliability analysis. The requirements for FMEA originate from the need to understand hardware behavior and its implications for the operation of the system or equipment. The need for FMEA can vary widely from one project to another. FMEA is the principal reliability engineering activity in support of the design review concept (see 4.2.1.4 of BS 5760: Part 1: 1988) and should be put into use from the very first steps of system and subsystem design. FMEA is applicable to all levels of system design but is most appropriate for lower levels where large numbers of items are involved and/or there is functional complexity. Special training of personnel performing FMEA is essential and they need the close collaboration of systems engineers and designers. The FMEA should be updated as the project progresses and as designs are modified. At the end of the project, FMEA is used to check the design and may be essential for demonstration of conformity of a designed system to the required standards, regulations, and user's requirements.

Information from the FMEA identifies priorities for statistical process control sampling and inspection tests during manufacture and installation and for qualification, approval, acceptance and start-up tests. It provides essential information for diagnostic and maintenance procedures for inclusion in handbooks.

In deciding on the extent and the way in which FMEA should be applied to an item or design, it is important to consider the specific purposes for which FMEA results are needed, the time phasing with other activities and the importance of establishing a predetermined degree of awareness and control over unwanted failure modes and effects. This leads to the planning of FMEA in qualitative terms at specified levels (system, subsystem, component, item) to relate to the iterative design and development process (see BS 5760: Part 1).

To ensure that it is effective, the place of FMEA should be clearly established in the reliability program, together with the time, manpower and other resources needed to make it effective. It is vital that FMEA is not abridged to save time and money. If time and money are short the FMEA should concentrate on those parts of the design which are new or are used in new ways. FMEA can be economically directed to areas identified as crucial by other methods of analysis, e.g. fault tree analysis (FTA).


\subsection{Uses of FMEA}

Some of the detailed applications and benefits of FMEA are listed below:

\begin{enumerate}
\item[(a)] to avoid costly modifications by the early identification of design deficiencies;
\item[(b)] to identify failures which, when they occur alone or in combination, have unacceptable or significant effects, and to determine the failure modes which may seriously affect the expected or required operation; \footnote{Such effects may include secondary failures.}
\item[(c)] to determine the need for the following:
\begin{enumerate}
\item[(1)] redundancy;
\item[(2)] design improvement;
\item[(3)] more generous stress allowances (derating);
\item[(4)] screening of items;
\item[(5)] design of features that ensure that the system fails in a preferred failure mode, e.g. 'fail-safe' outcomes of failures;
\item[(6)] selection of alternative materials, parts, devices, and components;
\end{enumerate}
\item[(d)] to identify serious failure consequences and hence the need for changes in design and/or operational rules;
\item[(e)] to provide the logic model required to evaluate the probability or rate of occurrence of anomalous operating conditions of the system in preparation for criticality analysis;
\item[(f)] to disclose safety hazard, and product liability problem areas, or non-compliance with regulatory requirements; \textbf{Note:} Frequently, separate studies will be required for safety, but overlap is inevitable and therefore cooperation is highly advisable.
\item[(g)] to ensure that the development test programme can detect potential failure modes;
\item[(h)] to focus upon key areas in which to concentrate quality control, inspection and manufacturing process controls;
\item[(i)] to assist in defining various aspects of the general maintenance strategy, such as:
\begin{enumerate}
\item[(1)] establishing the need for data recording and condition monitoring during testing, checking-out and use;
\item[(2)] provision of information for development of trouble-shooting guides;
\item[(3)] establishing maintenance cycles which anticipate and avoid wear-out failures;
\item[(4)] the selection of preventative or corrective maintenance schedules, facilities, equipment and staff;
\item[(5)] selection of built-in test equipment and suitable test points;
\end{enumerate}
\item[(j)] to provide a systematic and rigorous approach to the study of the installation in which the system is embedded;
\item[(k)] to facilitate or support the determination of test criteria, test plans and diagnostic procedures, for example: performance testing, reliability testing;
\item[(l)] to identify parts and assemblies requiring worst case analysis (frequently required for failure modes involving parameter drifts);
\item[(m)] to support the design of fault isolation sequences and to support the planning for alternative modes of operation and reconfiguration;
\item[(n)] to facilitate communication between the following:
\begin{enumerate}
\item[(1)] general and specialized engineers;
\item[(2)] equipment manufacturer and his suppliers;
\item[(3)] system user and the designer or manufacturer;
\end{enumerate}
\item[(o)] to enhance the analyst's knowledge and understanding of the behaviour of the equipment studied;
\item[(p)] to provide designers with an understanding of the factors which influence the reliability of the system;
\item[(q)] to provide a final document that is proof of the fact that (and of the extent to which) care has been taken to ensure that the design will meet its specification in service. (This is especially important in the case of product liability 
\end{enumerate}



\subsection{Limitations and Drawbacks}
FMEA is extremely efficient when it is applied to the analysis of elements that cause a failure of the entire system or of a major function of the system. However, FMEA may be difficult and tedious for the case of complex systems that have multiple functions involving different sets of system components. This is because of the quantity of detailed system information that needs to be considered. This difficulty can be increased by the existence of a number of possible operating modes, as well as by consideration of the repair and maintenance policies. FMEA can be a laborious and inefficient process unless it is judiciously applied. The uses to which the results are to be put subsequently should be defined and FMEA should not be included in requirements specifications indiscriminately. Complications, misunderstandings and errors can occur when FMEA attempts to span several levels in a hierarchical structure if redundancy is applied in the system design. It is therefore preferable for an FMEA to be restricted to relating two levels only in the hierarchical structure. For example, it is a relatively straightforward task to identify failure modes of items and to determine their effects on the assembly. These effects then become the failure modes at the next level up, e.g. the module, and so on. However, successful multi-level FMEAs are often carried out. FMEA is applicable to all levels of a system but is most appropriate to lower levels where large numbers of items are involved and/or there is functional complexity.

\subsection{Relationships with Other Methods}
FMEA (or FMECA) can be used alone. As a systematic inductive method of analysis, FMEA is most often used to complement other approaches, especially deductive ones. At the design stage, it is often difficult to decide whether the inductive or deductive approach is dominant, as both are combined in processes of thought and analysis. Where levels of risk are identified in industrial facilities and systems, the inductive approach is preferred and therefore FMEA is an essential design tool. However, it should be supplemented by other methods. This is particularly the case when problems need to be identified and solutions need to be found in situations where multiple failures and sequential effects need to be studied. The method used first will depend on the project programme.

During the early design stages, where only functions, general system structure and subsystems have been defined, successful performance of the system can be depicted by a reliability block diagram or a failure path by a fault tree. However, to assist in drawing these diagrams

\begin{figure}
        \includegraphics[width=\linewidth]{subfiles/imgs/fmea_criticality_grid.png}
  \caption{Criticality Grid}
        \label{fig:criticality_grid}
\end{figure}


\begin{figure}
        \includegraphics[width=\linewidth]{subfiles/imgs/fmea_criticality_grid.png}
  \caption{Criticality Grid}
        \label{fig:criticality_grid}
\end{figure}



\begin{figure}
	\includegraphics[width=\linewidth]{subfiles/imgs/fmea_criticality_grid.png}
  \caption{Criticality Grid}
	\label{fig:criticality_grid}
\end{figure}


\begin{figure}
	\includegraphics[width=\linewidth]{subfiles/imgs/fmea_criticality_matrix_with_bands.png}
  \caption{FMEA Criticality Matrix}
  \label{fig:criticality_matrix}
\end{figure}

\end{document}
