\documentclass[./dissertation.tex]{subfiles}
\begin{document}


\chapter{Problem, Proposals and Contributions}
The Failure Mode and Effects Analysis (FMEA) process is an essential tool for ensuring the safety and reliability of complex systems. It involves a systematic approach to identifying potential failure modes and their potential effects, as well as developing appropriate measures to prevent or mitigate these failures. However, despite its importance, the FMEA process is not without its challenges and limitations, especially in the context of certifying systems according to ISO26262.

One of the main challenges with the FMEA process is that it relies heavily on human expertise and experience. This means that the quality of the analysis can be highly dependent on the skills and knowledge of the individuals involved, which can lead to inconsistencies and errors. Additionally, the FMEA process can be time-consuming and resource-intensive, which can be problematic for organizations with limited budgets and schedules.

Another challenge with the FMEA process is that it can be difficult to capture and analyze all the potential failure modes and effects, especially in complex systems. This can lead to critical failure modes being missed or overlooked, which can have serious consequences for the safety and reliability of the system.

Moreover, the FMEA process is often conducted in isolation from other system development processes, which can lead to a lack of integration and coordination between different teams and departments. This can result in duplication of efforts, inconsistencies, and a failure to address all the potential failure modes and effects.

In addition to the challenges associated with the FMEA process, certifying a system according to ISO26262 also poses significant challenges. This standard is designed to ensure the safety of electronic systems used in road vehicles, and it requires a rigorous and systematic approach to system development and testing.

One of the main challenges with certifying a system according to ISO26262 is that it requires a significant amount of documentation and evidence to be produced to demonstrate compliance with the standard. This can be time-consuming and resource-intensive, and it requires a high level of expertise and knowledge.

Furthermore, ISO26262 requires a high level of integration and coordination between different teams and departments involved in the system development process. This can be challenging in large organizations with complex development processes, and it can require significant investment in training and development.

In conclusion, the FMEA process and certifying a system according to ISO26262 both pose significant challenges and require a high level of expertise, knowledge, and resources. However, these challenges can be overcome through careful planning, coordination, and a commitment to quality and safety. By addressing these challenges, organizations can ensure that their systems are safe, reliable, and compliant with industry standards and regulations.

\section{Timeliness and Relevance of the Subject}
The automation of the Failure Modes and Effects Analysis (FMEA) process is crucial due to the increasing complexity of systems, the need for enhanced efficiency, and the demand for more reliable products in today's fast-paced, technologically-driven world. This becomes even more relevant with the introduction of the ISO 26262 standard, which establishes stringent guidelines for the functional safety of electrical and electronic systems in the automotive industry. As systems become more intricate, the possibility of failure modes and their subsequent effects multiply, making manual FMEA methods laborious, time-consuming, and prone to human error. By leveraging artificial intelligence and advanced algorithms, automating the FMEA process allows for the swift identification of potential failure modes and their effects, ensuring a thorough and accurate analysis that complies with the rigorous requirements set forth by the ISO 26262 standard.

Furthermore, the automated FMEA process can reduce the workload on engineers and other experts by allowing them to focus on higher-level tasks and decision-making, while the system handles the minutiae of the analysis. This translates to substantial time and cost savings for organizations, which is essential for meeting the ISO 26262's requirement of achieving functional safety at an acceptable cost. Additionally, as the world becomes increasingly reliant on technology, the importance of product reliability and safety cannot be overstated. Automated FMEA not only improves the overall quality of products but also reduces the risk of catastrophic failures that can have severe consequences, both financially and in terms of human safety. This is particularly significant in the context of ISO 26262, as it aims to minimize the risk of systematic and random hardware failures in automotive systems.

In conclusion, automating the FMEA process is essential for the development of reliable, efficient, and safe products in an increasingly complex and competitive world. By adhering to the ISO 26262 standard, organizations can ensure that they are meeting the functional safety requirements for electrical and electronic systems in the automotive industry, and automated FMEA plays a crucial role in achieving this objective.


\section{Manuscript Organization}



\section{Contributions and Perspectives}


\end{document}
