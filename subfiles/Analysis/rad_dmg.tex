\section{Radiation Effects on CMOS Electronics}
\label{rad_effect_cmos}
It is necessary to discuss and understand the effects of radiation on CMOS electronics due to the highly radioactive environment of the CERN accelerators. This understanding will lead to understanding the necessity and the methods for radiation hardening of the electronics. The radiation affects the CMOS in two distinguishable ways. In the form of cumulative damages and single-event effects (SEEs). 

\subsection{Cumulative Damages}
Cumulative damages can be split into two subcategories. The first is non-ionizing processes, which come in the form of displacement of atoms in the lattice structure of the transistor. These are called displacement damages and are of little concern to CMOS technologies due to the high amount of doping \cite{giulioThesis}. The second is damages induced by ionizing doses, i.e. TID effects. MOS transistors can accumulate charges in the gate oxide, which creates a voltage difference on the gate and leads to unwanted biasing. This effect is dominant when the gate oxide is thick as it can contain a larger charge compared to the voltage threshold of the gate. Therefore, this effect decreases with smaller technologies as the gate oxide becomes thinner. The smaller technologies are instead dominated by effects like shallow trench isolation (STI) effects, which come in the form of radiation-induced drain-to-source leakage current and radiation-induced narrow channel effects (RINCE) \cite{giulioThesis}. Radiation-induced drain-to-source leakage current is caused by the accumulation of positive charges in the STI, which opens parasitic channels between the source and drain. This leads to an increase in leakage current. Positive charges are far more likely to be trapped due to electrons moving fast enough to leave the STI, while electron holes do not. However, over time electrons are attracted and enough electrons can be attracted to invert this effect. Therefore, initially, an increase in leakage current is observed, but as TID increases this effect reaches a peak and begins to invert. Since only positive charges are initially trapped, this effect does not increase the leakage current of pMOS. Instead, it repels the holes of the doped silicon increasing its threshold voltage and decreasing current flow. It is clear that increasing the length of the channel, decreases this effect as more charge is to be trapped before a channel can be opened. Therefore this effect is significant in smaller technologies with short gate lengths. However, this effect can be mitigated by using enclosed layout transistors (ELT), where the channel does not face the STI. 
%In figure -- a traditional layout of a transistor and a ELT layout can be seen. \todo{add figure}   

The other effect is RINCE, which is also due to the trapped charges in the STI. As positive charges are trapped in the STI, an electric field is created. This electric field leads to a decrease in threshold voltage for nMOS transistors and an increase in threshold voltage for pMOS transistors. However, as the width of the channel decrease, this effect becomes more dominant as the number of trapped charges does not change. This leads to a proportionally larger electric field, which signifies a dependency on channel width for this effect. For nMOS transistors, this effect is limited, as negative charges become trapped at the interface leading to the two canceling out similar to the inversion seen in radiation-induced drain-to-source leakage current at higher TID. However, for pMOS the trapped charges at the interface are also positive, leading to RINCE only increasing in potency with an increase in TID. As this effect is also due to charges trapped in the STI, it can be mitigated by the use of ELT \cite{giulioThesis}. However, these ELTs do use significantly more area compared to traditional designs. 

\subsection{Single-Event Effects}
Single-Event effects can be split into two categories, permanent single-event effects, and single-event upsets (SEUs) (and transients (SETs)). A permanent single-event is the possible creation of parasitic transistor structures between two n-wells, i.e. between two transistors. This can potentially shorten VDD and ground, which can permanently damage the device. However, this effect is limited due to highly doped substrates and the use of STI between wells \cite{aleThesis}. 

SEUs and transients are soft errors and not destructive to the die. Instead, they corrupt the information stored in digital logic circuits by flipping bits. SEUs become possible when the collected fraction of the charge liberated by an ionizing particle is larger than the electric charge stored on a sensitive node \cite{aleThesis}. This critical charge scales with the gate area of the design. As the gate area decreases, the amount of stored charge representing a logical value of information decreases. In general, SEU sensitivity is increased by the scaling down of technology as node capacitance and the supply voltage are both scaled down as well \cite{aleThesis}. 

A SET is an event, where a static combinatorial circuit is upset by a charged particle, leading to a glitch in the circuit. The time duration of SET is determined by the injected charge and the driving strength of the cell. If the output of this combinatorial circuit is sampled during the transient, a register can enter a metastable state, which can propagate through causing fatal errors \cite{aleThesis}. 
%Displacement damage
%TID dmgs
%
%
%Single event effects